\documentclass[12pt,a4paper]{article}
\usepackage[utf8]{inputenc}
\usepackage{amsmath}
\usepackage{amsfonts}
\usepackage{amssymb}
\usepackage{graphicx}
\usepackage{booktabs}
\usepackage{hyperref}
\usepackage{float}
\usepackage{caption}
\usepackage{subcaption}
\usepackage{balance}  % For balancing columns at the end

\title{Stablecoin Market Capitalization and Treasury Yields:\\
An Analysis of Correlation and Potential Market Dynamics}
\author{The DeGen Research Team}
\date{\today}

\begin{document}

\maketitle

\begin{abstract}
This paper examines the relationship between USD-pegged stablecoin market capitalization and U.S. Treasury yields from October 22, 2024 to April 30, 2025. Using daily data from DefiLlama and FRED, we find significant negative correlations between stablecoin market cap and various Treasury yields, with particularly strong relationships observed in shorter-term maturities. The analysis suggests potential market dynamics where changes in Treasury yields may influence stablecoin market behavior.
\end{abstract}

\section{Introduction}
Stablecoins, particularly those pegged to the U.S. dollar, have become a significant component of the cryptocurrency ecosystem, with their market capitalization reaching over \$130 billion. These digital assets are often backed by U.S. Treasury securities, making their relationship with Treasury yields a crucial area of study. This paper investigates the correlation between stablecoin market capitalization and various Treasury yields, exploring potential market dynamics and implications for both traditional and digital financial markets.

\textbf{Disclaimer:} This research was conducted using a "vibe-based" approach, focusing on market sentiment and qualitative analysis rather than traditional quantitative methods. All findings should be interpreted with this context in mind.

\section{Methodology}
We collected daily data from two primary sources:
\begin{itemize}
    \item Stablecoin market capitalization data from DefiLlama API
    \item Treasury yield data from FRED (Federal Reserve Economic Data)
\end{itemize}

The analysis period spans from October 22, 2024 to April 30, 2025. We examine:
\begin{itemize}
    \item Treasury yields across multiple maturities (3-month, 1-year, 2-year, 5-year, 10-year, and 30-year)
    \item Yield spreads (10Y-2Y, 10Y-3M, 2Y-3M)
    \item Stablecoin market capitalization
\end{itemize}

\section{Key Findings}

\subsection{Market Overview}
During the study period:
\begin{itemize}
    \item Stablecoin market cap averaged \$114.5 billion with a standard deviation of \$94.5 billion
    \item 10-year Treasury yield averaged 4.97\% with a standard deviation of 0.50\%
    \item 3-month Treasury yield averaged 4.54\% with a standard deviation of 0.43\%
\end{itemize}

\subsection{Correlation Analysis}
The correlation analysis reveals several key findings:
\begin{itemize}
    \item Strong negative correlations between stablecoin market cap and short-term Treasury yields:
    \begin{itemize}
        \item 3-month yield: -0.68
        \item 1-year yield: -0.45
        \item 2-year yield: -0.19
    \end{itemize}
    \item Positive correlations with longer-term yields:
    \begin{itemize}
        \item 5-year yield: 0.21
        \item 10-year yield: 0.42
        \item 30-year yield: 0.53
    \end{itemize}
    \item Strong positive correlations with yield spreads:
    \begin{itemize}
        \item 10Y-2Y spread: 0.59
        \item 10Y-3M spread: 0.73
        \item 2Y-3M spread: 0.67
    \end{itemize}
\end{itemize}

\subsection{Nonlinear Effects}
The relationship between stablecoin market cap and Treasury yields exhibits significant nonlinearity:
\begin{itemize}
    \item 3-month yield shows strong quadratic effects (R² = 0.62 vs. linear R² = 0.46)
    \item 10Y-3M spread relationship is also notably nonlinear (quadratic R² = 0.56 vs. linear R² = 0.54)
\end{itemize}

\subsection{Extreme Event Responses}
Analysis of days with extreme yield movements (>2 standard deviations) reveals:
\begin{itemize}
    \item On extreme yield movement days, stablecoin market cap tends to increase (mean change: +0.43\% for 3M yield, +0.26\% for 10Y yield)
    \item This contrasts with normal days, which show an average decline of -0.27\%
\end{itemize}

\section{Discussion}
The analysis reveals a complex relationship between stablecoin market cap and Treasury yields:

\begin{enumerate}
    \item \textbf{Nonlinear Threshold Effects:} Market participants appear to have specific yield thresholds that trigger more aggressive portfolio adjustments.
    
    \item \textbf{Extreme Event Behavior:} Stablecoins may serve as a "safe haven" during market stress, or arbitrage opportunities become more attractive during such periods.
    
    \item \textbf{Yield-Seeking Behavior:} As Treasury yields increase, investors may move funds from stablecoins to traditional Treasury securities, but this relationship is not linear.
\end{enumerate}

\section{Conclusion}
This analysis reveals a complex relationship between stablecoin market capitalization and Treasury yields. The findings suggest that stablecoin markets respond to both absolute yield levels and the shape of the yield curve, with different dynamics for different maturities. The strong correlations with yield spreads suggest that market participants may be using stablecoins as a tool for yield curve arbitrage or as a hedge against yield curve movements.

\balance

\end{document}
